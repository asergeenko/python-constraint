\documentclass[a4paper]{article}

\usepackage[english]{babel}
\usepackage[utf8]{inputenc}

\usepackage{algorithm}
\usepackage[noend]{algpseudocode}

\usepackage{titlesec}
\newcommand{\sectionbreak}{\clearpage}

\begin{document}
\title{Solving constraint satisfaction problems using arc consistency and path consistency techniques}
\maketitle
\section{Local consistency algorithms}
\subsection{Arc consistency}
For the arc consistency we use AC1 algorithm.
\begin{algorithm}
    \caption{AC1 algorithm}
    \begin{algorithmic}[1]
        \Procedure{Revise}{$v_i,v_j$}\Comment{a network with two variables $v_i$,$v_j$, domains $D_i$ and $D_j$, and constraint $R_{ij}$ }
            \For {each $ a_i\in D_i$}
                \If {there is no $a_j \in D_j$ with $(a_i,a_j) \in R_{ij}$}
                    \State {remove $a_i$ from $D_i$}
                \EndIf
            \EndFor
        \EndProcedure
    \end{algorithmic}
    \begin{algorithmic}[2]
        \Procedure{AC1}{$N$}\Comment{a constraint network $N = <V,D,C>$}
            \Repeat
                \For {each arc $(v_i, v_j)$ with $R _{ij}\in C$}
                    \State{Revise$(v_i,v_j)$}
                    \State{Revise$(v_j,v_i)$}
                \EndFor
            \Until{no domain is changed}
        \EndProcedure
    \end{algorithmic}
\end{algorithm}

\subsection{Path consistency}
For the path consistency we use PC1 algorithm.
\begin{algorithm}
    \caption{PC1 algorithm}
    \begin{algorithmic}[1]
        \Procedure{Path\_Revise}{$\{v_i,v_j\},v_k$}\Comment{a binary network $<V,D,C>$ with variables $v_i$,$v_j$,$v_k$}
            \For {each pair $ (a_i,a_j)\in R_{ij}$}
                 \If {there is no $a_k \in D_k$ such that $(a_i,a_k) \in R_{ik}$ and $(a_j,a_k) \in R_{jk}$} \State {remove $(a_i,a_j)$ from $R_{ij}$}
                \EndIf
            \EndFor
        \EndProcedure
    \end{algorithmic}
    
    \begin{algorithmic}[2]
        \Procedure{PC1}{$N$}\Comment{a constraint network $N = <V,D,C>$}
            \Repeat
                \For {each (ordered) triple of variables $v_i, v_j, v_k$}
                    \State{Path\_Revise$({v_i,v_j},v_k)$}
                \EndFor
            \Until{no constraint is changed}
        \EndProcedure
    \end{algorithmic}
\end{algorithm}

\section{Python implementation}
For the implementation we use modified \textbf{python\_constraint} module.
\subsection{Installation}
\begin{itemize}
    \item Clone the repository \newline \texttt{git clone https://github.com/asergeenko/python-constraint.git}
    \item Build and setup the module \newline \texttt{python setup.py build}\newline \texttt{python setup.py install}
\end{itemize}
\subsection{Module overview}
Python implementations of the AC1 and PC1 algorithms are located in \texttt{consistency.py} file. There are four methods:\\\\
\textbf{arc\_revise}\textit{(var1,var2,problem,constraints\_for\_variable)}\\
\textit{\#Implements REVISE procedure from AC1 algorithm}\\

\textbf{Input:}

\textit{var1} and \textit{var2} - variables to revise,

\textit{problem} - \textbf{python\_constraint}'s \textit{Problem} instance,

\textit{constraints\_for\_variable} - dictionary with constraints for variable \textit{var1}

(returned by \textit{getArcs} function from \textbf{constraint} module).\\


\textbf{Output:}

Returns \textit{True} if arc \textit{(var1,var2)} is not arc-consistent

and some values are removed from the domain.\\\\
\textbf{ac1}\textit{(arcs, problem)}\\
\textit{\#Implements AC1 procedure}\\
 
\textbf{Input:}

\textit{arcs} - arcs (returned by \textit{getArcs} function from \textbf{constraint} module),

\textit{problem} - \textbf{python\_constraint}'s \textit{Problem} instance\\\\ 
\textbf{path\_revise}\textit{(var1, var2, var3,problem)}\\
\textit{\#Implements PATH\_REVISE procedure from PC1 algorithm}\\


\textbf{Input:}

\textit{var1},\textit{var2}.\textit{var3} - variables to revise,

\textit{problem} - \textbf{python\_constraint}'s \textit{Problem} instance\\


\textbf{Output:}

Returns \textit{True} if the pair \textit{(var1,var2)} is not path-consistent relative to \textit{var3}

and some values are removed.\\\\
\textbf{pc1}\textit{(problem)}\\
\textit{\#Implements PC1 procedure}\\


\textbf{Input:}

\textit{problem} - \textbf{python\_constraint}'s \textit{Problem} instance\\
 
\section{Problem solving examples}
\subsection{Examples overview}
These algorithms are suitable for CSP problems with binary constraints. So we have \textit{N-Queens problem} (can be found in \texttt{examples/queens/queens.py}) and \textit{Map coloring problem} (can be found in \texttt{examples/map\_coloring/map\_coloring.py}).
N-queens example is the part of the original \textbf{python\_constraint} module and has just one parameter:\\

\texttt{size = 8} \textit{\# size of the chessboard}\\\\
Map coloring example is new and has the following parameters:\\

\texttt{countries = ['A','B','C']} \textit{\# countries}

\texttt{num\_colors = 2} \textit{\# number of colors}

\texttt{neighbors = ['AB','BC','CA']} \textit{\# countries that border each other}
\subsection{Command-line parameters}

\texttt{python queens.py [-h] [-s]\footnote{This parameter is used only with \texttt{queens.py} example.} [-ac] [-pc]}\\
\texttt{python map\_coloring.py [-h] [-ac] [-pc]}\\\\
\texttt{-h} - show help message\\
\texttt{-s} - show solutions on the chessboard (doesn't show by default)\\
\texttt{-ac} - use arc consistency algorithm (not used by default)\\
\texttt{-pc} - use path consistency algorithm (not used by default)\\
\end{document}

